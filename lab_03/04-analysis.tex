\chapter{Аналитический раздел}

Сортировкой называют перестановку объектов, при которой они располагаются в порядке возрастания или убывания \cite{knut}.

В данном разделе будут описаны три алгоритма сортировок: блочная, быстрая и выбором.

\section{Алгоритм блочной сортировки}
Данный алгоритм можно разделить на следующие шаги \cite{blocksort}:

\begin{itemize}
	\item разделить массив на две равные (или почти равные) части;
	\item отсортировать каждую часть массива отдельно путем рекурсивного применения блочной сортировки;
	\item слить отсортированные части массива в один отсортированный массив путем сравнения элементов и добавления их в правильном порядке;
\end{itemize}


\section{Алгоритм быстрой сортировки}
Данный алгоритм можно разделить на следующие шаги \cite{quicksort}:

\begin{itemize}
	\item разбить массива относительно опорного элемента;
	\item рекурсивно отсортировать каждую часть массива:
\end{itemize}


\section{Алгоритм сортировки выбором}
Данный алгоритм можно разделить на следующие шаги \cite{selectionsort}:

\begin{itemize}
	\item взять первый элемент последовательности A[i], здесь i – номер элемента, для первого i равен 1;
	\item найти минимальный (максимальный) элемент последовательности и запомнить его номер в переменную key;
	\item если номер первого элемента и номер найденного элемента не совпадают, т. е. если key != 1, тогда два этих элемента обменять значениями;
	\item увеличить i на 1 и продолжить сортировку оставшейся части массива, а именно с элемента с номером 2 по N, так как элемент A[1] уже занимает свою позицию.
\end{itemize}


\section*{Вывод}

В данном разделе были описаны три алгоритма сортировок: блочная, быстрая и выбором.
