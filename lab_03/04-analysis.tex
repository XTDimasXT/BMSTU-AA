\chapter{Аналитический раздел}

Сортировкой называют перестановку объектов, при которой они располагаются в порядке возрастания или убывания \cite{knut}.

В данном разделе будут описаны три алгоритма сортировок: блочная, быстрая и выбором.

\section{Алгоритм блочной сортировки}
Идея заключается в  разбиении входных данных на <<блоки>> одинакового размера, после чего данные в блоках сортируются и результаты сортировок объединяются.
Отсортированная последовательность получается путём последовательного перечисления элементов каждого блока.
Для деления данных на блоки, алгоритм предполагает, что значения распределены равномерно, и распределяет элементы по блокам равномерно. 
Например, предположим, что данные имеют значения в диапазоне от 1 до 100 и алгоритм использует 10 блоков. 
Алгоритм помещает элементы со значениями 1--10 в первый блок, со значениями 11--20  во второй, и т.д.
Если элементы распределены равномерно, в каждый блок попадает примерно одинаковое число элементов. 
Если в списке $N$ элементов, и алгоритм использует $N$ блоков, в каждый блок попадает всего один или два элемента, поэтому возможно отсортировать элементы за конечное число шагов \cite{article_sorts}.


\section{Алгоритм быстрой сортировки}
Данный алгоритм можно разделить на следующие шаги \cite{quicksort}:

\begin{itemize}
	\item разбить массива относительно опорного элемента;
	\item рекурсивно отсортировать каждую часть массива:
\end{itemize}


\section{Алгоритм сортировки выбором}
Данный алгоритм можно разделить на следующие шаги \cite{selectionsort}:

\begin{itemize}
	\item взять первый элемент последовательности A[i], здесь i – номер элемента, для первого i равен 1;
	\item найти минимальный (максимальный) элемент последовательности и запомнить его номер в переменную key;
	\item если номер первого элемента и номер найденного элемента не совпадают, т. е. если key != 1, тогда два этих элемента обменять значениями;
	\item увеличить i на 1 и продолжить сортировку оставшейся части массива, а именно с элемента с номером 2 по N, так как элемент A[1] уже занимает свою позицию.
\end{itemize}


\section*{Вывод}

В данном разделе были описаны три алгоритма сортировок: блочная, быстрая и выбором.
