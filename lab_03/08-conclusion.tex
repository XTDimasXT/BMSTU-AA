\chapter*{ЗАКЛЮЧЕНИЕ}
\addcontentsline{toc}{chapter}{ЗАКЛЮЧЕНИЕ}

Цель данной лабораторной работы была достигнута, а именно были исследованы алгоритмы сортировок.

Для достижения поставленной цели были выполнены следующие задачи.
\begin{itemize}
	\item описаны алгоритмы блочной, быстрой сортировок и сортировки выбором;
	\item разработано программное обеспечение, реализующее алгоритмы сортировок;
	\item выбраны инструменты для реализации алгоритмов и замера процессорного времени их выполнения;
	\item проведен анализ затрат реализаций алгоритмов по времени и памяти. 
\end{itemize}

В результате исследования реализаций различных алгоритмов было получено, что для массивов длиной 1000, отсортированных в обратном порядке, реализация алгоритма быстрой сортировки по времени оказалась в 2.59 раз лучше, чем реализация блочной сортировки, и в 15.49 раз лучше реализации сортировки выбором.
В свою очередь, реализация блочной сортировки оказалась лучше в 5.99 раз по времени выполнения, чем реализация  сортировки выбором.

Для отсортированных массивов длиной 1000 реализация быстрой сортировки оказалась лучше по времени в 2.39 раз, чем реализация блочной сортировки, и в 15.6 раз лучше, чем реализация сортировки выбором. 
В свою очередь, реализация блочной сортировки оказалась лучше в 6.52 раза по времени выполнения, чем реализация сортировки выбором.

Для случайно упорядоченных массивов длиной 1000 реализация быстрой сортировки оказалась лучше по времени в 2.49 раз, чем реализация блочной сортировки, и в 15.01 раза лучше, чем реализация сортировки выбором. 
В свою очередь, реализация блочной сортировки на случайно упорядоченных массивах оказалась лучше в 6.03 раза по времени выполнения, чем реализация сортировки выбором.

Для массивов длиной менее 300, реализация блочной сортировки была лучше или такой же по времени выполнения по сравнению с быстрой сортировкой, но на массивах длиной более 300 быстрая сортировка становилась лучше по времени выполнению.

В результате теоретической оценки алгоритмов по памяти можно сделать вывод о том, что алгоритм сортировки выбором является наименее ресурсозатратным.
Алгоритм быстрой сортировки, напротив, требует больше всего памяти, что объясняется тем, что алгоритм рекурсивный, и на каждый вызов функции требуется выделение памяти на стеке для сохранения информации, связанной с этим вызовом.