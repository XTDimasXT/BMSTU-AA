\chapter*{ЗАКЛЮЧЕНИЕ}
\addcontentsline{toc}{chapter}{ЗАКЛЮЧЕНИЕ}

Из результатов проведения лабораторной работы и исследования можно сделать вывод, что временные затраты всех реализаций алгоритмов нахождения расстояний Левенштейна и Дамерау---Левенштейна растут в зависимости от длины строки. Рекурсивная реализация алгоритма поиска расстояния Дамерау-Левенштейна выигрывает остальные реализации в затрачиваемой памяти, но проигрывает по процессорному времени. Матричные реализации затрачивают наименьшее количество процессорного времени.

Была достигнута цель лабораторной работы и выполнены все задачи, а именно были описаны, изучены и сравнены несколько алгоритмов поиска редакционного расстояния.

Для достижения поставленной цели были выполнены следующие задачи:
\begin{itemize}
	\item описаны расстояния Левенштейна и Дамерау---Левенштейна;
	\item создан программный продукт с реализованными алгоритмами поиска расстояний Левенштейна и Дамерау---Левенштейна;
	\item исследованы затраты времени и памяти при различных реализациях алгоритмов;
	\item выполнено сравнение алгоритмов по процессорному времени и затратам оперативной памяти.
\end{itemize}