\chapter{Технологический раздел}

Для создания программного продукта был выбран язык Python~\cite{python}.
Данный выбор обусловлен следующим:
\begin{itemize}
	\item язык поддерживает все структуры данных, которые выбраны в результате проектирования;
	\item язык позволяет реализовать все алгоритмы, выбранные в результате проектирования;
	\item язык позволяет замерять процессорное время с помощью модуля \textit{time}. 
\end{itemize}

Для разработки была выбрана среда разработки Visual Studio Code в связи с тем, что она предоставляет возможности управления и ведения проекта для выполнения поставленной задачи~\cite{vscode}.

\section{Реализация алгоритмов}

На листингах \ref{lst:lev.py}--\ref{lst:dam_lev_cache.py} представлены реализации алгоритмов поиска расстояний Левенштейна и Дамерау---Левенштейна.

\newpage

\includelistingpretty
	{lev.py}
	{python}
	{Реализация матричного алгоритма поиска редакционного расстояния Левенштейна}

\newpage

\includelistingpretty
	{dam_lev.py}
	{python}
	{Реализация матричного алгоритма поиска редакционного расстояния Дамерау---Левенштейна}

\newpage

\includelistingpretty
	{dam_lev_rec.py}
	{python}
	{Реализация рекурсивного алгоритма поиска редакционного расстояния Дамерау---Левенштейна}

\newpage

\includelistingpretty
	{dam_lev_cache.py}
	{python}
	{Реализация рекурсивного алгоритма поиска редакционного расстояния Дамерау---Левенштейна с кэшированием}

\section{Тестовые случаи}

Обозначения:
\begin{itemize}
	\item Lev --- матричная реализация алгоритма поиска расстояния Левенштейна;
	\item Dam-Lev --- матричная реализация алгоритма поиска расстояния Дамерау-Левенштейна;
	\item Rec --- рекурсивная реализация алгоритма поиска расстояния Дамерау-Левенштейна;
	\item Rec-Cache --- рекурсивная реализация алгоритма поиска расстояния Дамерау-Левенштейна с кэшированием.
\end{itemize}

В таблице \ref{table:tests} приведены тестовые случаи.

\begin{table}[H]
	\centering
	\caption{\label{table:tests} Тестовые случаи}
	\begin{center}
		\begin{tabular}{|r|r|r|r|r|r|r|}
			\hline № & $S_1$ & $S_2$ & Lev & Dam-Lev & Rec & Rec-Cache \\ \hline
			1 & привет & пока & 5 & 5 & 5 & 5 \\ \hline
			2 & poka & privet & 5 & 5 & 5 & 5 \\ \hline
			3 & привет & пирвет & 2 & 1 & 1 & 1 \\ \hline
			4 & коммуникация & комуна & 6 & 6 & 6 & 6 \\ \hline
			5 & wife & husband & 7 & 7 & 7 & 7 \\ \hline
			6 & sunday & saturday & 3 & 3 & 3 & 3 \\ \hline
			7 & John & Jhno & 2 & 2 & 2 & 2 \\ \hline
		\end{tabular}
	\end{center}
\end{table}