\chapter*{ВВЕДЕНИЕ}

Зачастую при использовании поисковых платформ в спешке пользователи допускают ошибки в написании слов. 
После обработки поискового запроса пользователь получает предложение об исправлении.
Метрика, реализующая поиск подобных исправлений, называется редакционным расстоянием.  
Очевидно, что ее применение не заканчивается на этом. 
Проблема посимвольного сравнения при работе со словами встречается повсеместно.

Впервые задачу определения редакционного расстояния поставил советский математик Владимир Левенштейн в 1965 году во время изучения последовательностей 0--1. 
Стоит заметить, что более общую задачу назвали его именем. 
Алгоритм расстояния Левенштейна допускает следующие операции: вставка, удаление и замена символа.

На данный момент расстояние Левенштейна активно применяется:
\begin{itemize}
	\item для сравнения текстовых файлов утилитой ${diff}$;
	\item для исправления ошибок в слове в поисковых системах, базах данных, системах автоматического распознавания сканированного текста или речи;
	\item в биоинформатике для сравнения генов, хромосом и белков.
\end{itemize}

Позднее Фредерик Дамерау обнаружил, что зачастую ошибки связаны с неправильным порядком записи соседних букв, и добавил операцию транспозиции (перестановки) символов.

Целью данной лабораторной работы является описание, изучение и сравнение нескольких алгоритмов поиска редакционного расстояния.

Для выполнения поставленной цели необходимо выполнить следующие задачи:
\begin{itemize}
	\item описать расстояние Левенштейна и Дамерау---Левенштейна;
	\item создать программный продукт с реализованными алгоритмами поиска расстояний Левенштейна и Дамерау---Левенштейна;
	\item исследовать затраты времени и памяти при различных реализациях алгоритмов;
	\item выполнить сравнение алгоритмов по затратам процессорного времени и памяти.
\end{itemize}

\addcontentsline{toc}{chapter}{ВВЕДЕНИЕ}