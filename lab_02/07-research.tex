\chapter{Исследовательский раздел}

В данном разделе будут приведены примеры работы программ, постановка эксперимента и сравнительный анализ алгоритмов на основе полученных данных.

\section{Демонстрация работы программы}


На рисунке \ref{img:program} представлена демонстрация работы разработанного программного обеспечения, а именно показаны результаты умножения матриц $A = \begin{pmatrix}
	1 & 2 & 3\\
	4 & 5 & 6 \\
	7 & 8 & 9\\
\end{pmatrix}$ и $B = \begin{pmatrix}
	3 & 6 & 7\\
	2 & 5 & 8 \\
	1 & 4 & 9 \\
\end{pmatrix}$.  
\clearpage

\includeimage
{program} % Имя файла без расширения (файл должен быть расположен в директории inc/img/)
{f} % Обтекание (без обтекания)
{h} % Положение рисунка (см. figure из пакета float)
{1\textwidth} % Ширина рисунка
{Демонстрация работы программы при умножении двух матриц} % Подпись рисунка

\clearpage


\section{Технические характеристики}

Технические характеристики компьютера, на котором выполнялись замеры по времени.

\begin{itemize}
	\item процессор Intel Core i5-10400F (6 ядер) \cite{intel};
	\item 16 Гб оперативная память DDR4;
	\item операционная система Windows 10 Pro \cite{windows}.
\end{itemize}

Во время проведения исследования компьютер был нагружен только системными приложениями и целевой программой.

\section{Время выполнения реализаций алгоритмов}

Результаты замеров времени выполнения реализаций алгоритмов умножения матриц приведены в таблицах \ref{tbl:time_measurements} -- \ref{tbl:time_measurements_odd}.
Замеры времени проводились на квадратных матрицах одного порядка и усреднялись для каждого набора одинаковых экспериментов.
В таблицах \ref{tbl:time_measurements} -- \ref{tbl:time_measurements_odd} используются следующие обозначения: 
\begin{itemize}
	\item К --- реализация классического алгоритма умножения матриц;
	\item В --- реализация алгоритма Винограда умножения матриц;
	\item ВО --- реализация алгоритма Винограда с оптимизацией умножения матриц;
	\item Ш --- реализация алгоритма Штрассена умножения матриц.
\end{itemize}

\begin{table}[h]
	\begin{center}
		\begin{threeparttable}
			\captionsetup{justification=raggedright,singlelinecheck=off}
			\caption{Время работы реализации алгоритмов (в с)}
			\label{tbl:time_measurements}
			\begin{tabular}{|c|c|c|c|c|}
				\hline
				Порядок матриц & К & В & ВО & Ш \\
				\hline
				6 &$ 1.88\cdot 10^{-4} $&$ 1.88\cdot 10^{-4} $&$ 1.56\cdot 10^{-4} $&$ 1.31\cdot 10^{-3}$\\
				\hline
				11 &$ 9.06\cdot 10^{-4} $&$ 1.12\cdot 10^{-3} $&$ 9.37\cdot 10^{-4} $&$ 9.16\cdot 10^{-3}$\\
				\hline
				16 &$ 2.78\cdot 10^{-3} $&$ 3.28\cdot 10^{-3} $&$ 2.47\cdot 10^{-3} $&$ 9.28\cdot 10^{-3}$\\
				\hline
				21 &$ 6.47\cdot 10^{-3} $&$ 6.84\cdot 10^{-3} $&$ 5.25\cdot 10^{-3} $&$ 6.48\cdot 10^{-2}$\\
				\hline
				26 &$ 1.14\cdot 10^{-2} $&$ 1.27\cdot 10^{-2} $&$ 1.03\cdot 10^{-2} $&$ 6.56\cdot 10^{-2}$\\
				\hline
				31 &$ 2.00\cdot 10^{-2} $&$ 2.20\cdot 10^{-2} $&$ 1.71\cdot 10^{-2} $&$ 6.54\cdot 10^{-2}$\\
				\hline
				36 &$ 3.09\cdot 10^{-2} $&$ 3.40\cdot 10^{-2} $&$ 2.56\cdot 10^{-2} $&$ 4.60\cdot 10^{-1}$\\
				\hline
				41 &$ 4.62\cdot 10^{-2} $&$ 5.13\cdot 10^{-2} $&$ 3.77\cdot 10^{-2} $&$ 4.53\cdot 10^{-1}$\\
				\hline
				46 &$ 6.29\cdot 10^{-2} $&$ 7.06\cdot 10^{-2} $&$ 5.40\cdot 10^{-2} $&$ 4.77\cdot 10^{-1}$\\
				\hline
				51 &$ 8.78\cdot 10^{-2} $&$ 9.47\cdot 10^{-2} $&$ 7.37\cdot 10^{-2} $&$ 4.54\cdot 10^{-1}$\\
			\hline
			\end{tabular}
		\end{threeparttable}
	\end{center}
\end{table}

\begin{table}[h]
	\begin{center}
		\begin{threeparttable}
			\captionsetup{justification=raggedright,singlelinecheck=off}
			\caption{Время работы реализации алгоритмов для четных порядков матриц (в с)}
			\label{tbl:time_measurements_even}
			\begin{tabular}{|c|c|c|c|c|}
				\hline
				Порядок матриц & К & В & ВО & Ш \\
				\hline
				6 &$ 1.88\cdot 10^{-4} $&$ 1.88\cdot 10^{-4} $&$ 1.56\cdot 10^{-4} $&$ 1.37\cdot 10^{-3}$\\
				\hline
				10 &$ 7.19\cdot 10^{-4} $&$ 8.12\cdot 10^{-4} $&$ 6.25\cdot 10^{-4} $&$ 9.31\cdot 10^{-3}$\\
				\hline
				14 &$ 1.84\cdot 10^{-3} $&$ 2.16\cdot 10^{-3} $&$ 1.75\cdot 10^{-3} $&$ 9.16\cdot 10^{-3}$\\
				\hline
				18 &$ 3.87\cdot 10^{-3} $&$ 4.38\cdot 10^{-3} $&$ 3.34\cdot 10^{-3} $&$ 6.59\cdot 10^{-2}$\\
				\hline
				22 &$ 7.16\cdot 10^{-3} $&$ 7.78\cdot 10^{-3} $&$ 6.09\cdot 10^{-3} $&$ 6.63\cdot 10^{-2}$\\
				\hline
				26 &$ 1.14\cdot 10^{-2} $&$ 1.29\cdot 10^{-2} $&$ 1.04\cdot 10^{-2} $&$ 6.62\cdot 10^{-2}$\\
				\hline
				30 &$ 1.83\cdot 10^{-2} $&$ 1.96\cdot 10^{-2} $&$ 1.53\cdot 10^{-2} $&$ 6.70\cdot 10^{-2}$\\
				\hline
				34 &$ 2.58\cdot 10^{-2} $&$ 2.82\cdot 10^{-2} $&$ 2.13\cdot 10^{-2} $&$ 4.53\cdot 10^{-1}$\\
				\hline
				38 &$ 3.51\cdot 10^{-2} $&$ 3.90\cdot 10^{-2} $&$ 3.00\cdot 10^{-2} $&$ 4.51\cdot 10^{-1}$\\
				\hline
				42 &$ 4.73\cdot 10^{-2} $&$ 5.20\cdot 10^{-2} $&$ 4.10\cdot 10^{-2} $&$ 4.77\cdot 10^{-1}$\\
				\hline
				46 &$ 6.45\cdot 10^{-2} $&$ 6.99\cdot 10^{-2} $&$ 5.23\cdot 10^{-2} $&$ 4.71\cdot 10^{-1}$\\
				\hline
				50 &$ 7.98\cdot 10^{-2} $&$ 8.73\cdot 10^{-2} $&$ 6.86\cdot 10^{-2} $&$ 4.78\cdot 10^{-1}$\\
				\hline
			\end{tabular}
		\end{threeparttable}
	\end{center}
\end{table}

\begin{table}[h]
	\begin{center}
		\begin{threeparttable}
			\captionsetup{justification=raggedright,singlelinecheck=off}
			\caption{Время работы реализации алгоритмов для нечетных порядков матриц (в с)}
			\label{tbl:time_measurements_odd}
			\begin{tabular}{|c|c|c|c|c|}
				\hline
				Порядок матриц & К & В & ВО & Ш \\
				\hline
				7 &$ 2.50\cdot 10^{-4} $&$ 3.13\cdot 10^{-4} $&$ 2.50\cdot 10^{-4} $&$ 1.34\cdot 10^{-3}$\\
				\hline
				11 &$ 9.06\cdot 10^{-4} $&$ 1.12\cdot 10^{-3} $&$ 9.06\cdot 10^{-4} $&$ 9.78\cdot 10^{-3}$\\
				\hline
				15 &$ 2.37\cdot 10^{-3} $&$ 2.66\cdot 10^{-3} $&$ 2.06\cdot 10^{-3} $&$ 9.81\cdot 10^{-3}$\\
				\hline
				19 &$ 4.75\cdot 10^{-3} $&$ 5.22\cdot 10^{-3} $&$ 4.09\cdot 10^{-3} $&$ 6.54\cdot 10^{-2}$\\
				\hline
				23 &$ 7.94\cdot 10^{-3} $&$ 9.19\cdot 10^{-3} $&$ 7.16\cdot 10^{-3} $&$ 6.49\cdot 10^{-2}$\\
				\hline
				27 &$ 1.34\cdot 10^{-2} $&$ 1.49\cdot 10^{-2} $&$ 1.08\cdot 10^{-2} $&$ 6.53\cdot 10^{-2}$\\
				\hline
				31 &$ 1.92\cdot 10^{-2} $&$ 2.14\cdot 10^{-2} $&$ 1.67\cdot 10^{-2} $&$ 6.50\cdot 10^{-2}$\\
				\hline
				35 &$ 2.79\cdot 10^{-2} $&$ 3.07\cdot 10^{-2} $&$ 2.36\cdot 10^{-2} $&$ 4.59\cdot 10^{-1}$\\
				\hline
				39 &$ 3.87\cdot 10^{-2} $&$ 4.32\cdot 10^{-2} $&$ 3.32\cdot 10^{-2} $&$ 4.57\cdot 10^{-1}$\\
				\hline
				43 &$ 5.26\cdot 10^{-2} $&$ 5.77\cdot 10^{-2} $&$ 4.47\cdot 10^{-2} $&$ 4.79\cdot 10^{-1}$\\
				\hline
				47 &$ 6.85\cdot 10^{-2} $&$ 7.50\cdot 10^{-2} $&$ 5.71\cdot 10^{-2} $&$ 4.50\cdot 10^{-1}$\\
				\hline
				51 &$ 8.81\cdot 10^{-2} $&$ 9.60\cdot 10^{-2} $&$ 7.37\cdot 10^{-2} $&$ 4.81\cdot 10^{-1}$\\
			\hline
			\end{tabular}
		\end{threeparttable}
	\end{center}
\end{table}

\clearpage
На рисунках \ref{img:figure} -- \ref{img:figure_odd} изображены графики зависимостей времени выполнения реализаций алгоритмов от порядка умножаемых матриц.

\includeimage
{figure} % Имя файла без расширения (файл должен быть расположен в директории inc/img/)
{f} % Обтекание (без обтекания)
{h} % Положение рисунка (см. figure из пакета float)
{1\textwidth} % Ширина рисунка
{Сравнение реализаций алгоритмов по времени выполнения} % Подпись рисунка

\includeimage
{figure_even} % Имя файла без расширения (файл должен быть расположен в директории inc/img/)
{f} % Обтекание (без обтекания)
{h} % Положение рисунка (см. figure из пакета float)
{1\textwidth} % Ширина рисунка
{Сравнение реализаций алгоритмов по времени выполнения для четных порядков матриц} % Подпись рисунка

\includeimage
{figure_odd} % Имя файла без расширения (файл должен быть расположен в директории inc/img/)
{f} % Обтекание (без обтекания)
{h} % Положение рисунка (см. figure из пакета float)
{1\textwidth} % Ширина рисунка
{Сравнение реализаций алгоритмов по времени выполнения для нечетных порядков матриц} % Подпись рисунка


\clearpage

\section*{Вывод}

В результате замеров времени выполнения реализаций различных алгоритмов было получено, что для матриц с порядком 51, реализация алгоритма Винограда оказалась в 1.08 раз хуже реализации классического алгоритма по времени выполнения, при этом реализация алгоритма Винограда с оптимизациями оказалась лучше в 1.19 раз реализации классического алгоритма и в 1.28 раз лучше реализации без оптимизаций по времени выполнения. 
Для матриц с четным порядком 50, реализация алгоритма Винограда хуже реализации классического алгоритма в 1.09 раз по времени выполнения, а реализация алгоритма Винограда с оптимизацией оказалась лучше классического алгоритма в 1.16 раз. 

Для матриц с порядком 51, реализация алгоритма Штрассена оказалась хуже по времени выполнения в 5.17 раз, в 4.79 раз и в 6.16 раз, чем реализации классического алгоритма, алгоритма Винограда и алгоритма Винограда с оптимизациями соответственно. 
Такой результат обусловлен тем, что в данной реализации много времени тратится на выделение подматриц, их сложение и вычитание, а также на рекурсивные вызовы. По теоретической оценке трудоемкости, можно заметить насколько велика константа перед $n^{\log_{2}7}$.

Стоит заметить, что на графике зависимости времени выполнения реализаций алгоритмов от порядка матриц \ref{img:figure}, у реализации алгоритма Штрассена присутствуют отчетливые ступеньки.
Это обусловлено особенностью алгоритма, который может работать только с квадратными матрицами порядка $2^k$, а матрицы других размеров приводятся к данному размеру дополнением нулевыми строками и столбцами. 
Таким образом, разные размеры $n$ могу приводится к одному новому порядку $2^k$, что в свою очередь приводит к почти идентичному времени обработки.

Полученные на практике результаты примерно соответствуют теоретическим оценкам.