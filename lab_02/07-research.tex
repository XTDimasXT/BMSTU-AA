\chapter{Исследовательский раздел}

В данном разделе будут приведены примеры работы программ, постановка эксперимента и сравнительный анализ алгоритмов на основе полученных данных.

\section{Демонстрация работы программы}


На рисунке \ref{img:program} представлена демонстрация работы разработанного программного обеспечения, а именно показаны результаты умножения матриц $A = \begin{pmatrix}
	1 & 2 & 3\\
	4 & 5 & 6 \\
	7 & 8 & 9\\
\end{pmatrix}$ и $B = \begin{pmatrix}
	3 & 6 & 7\\
	2 & 5 & 8 \\
	1 & 4 & 9 \\
\end{pmatrix}$.  
\clearpage

\includeimage
{program} % Имя файла без расширения (файл должен быть расположен в директории inc/img/)
{f} % Обтекание (без обтекания)
{h} % Положение рисунка (см. figure из пакета float)
{1\textwidth} % Ширина рисунка
{Демонстрация работы программы при умножении двух матриц} % Подпись рисунка

\clearpage


\section{Технические характеристики}

Технические характеристики компьютера, на котором выполнялись замеры по времени.

\begin{itemize}
	\item процессор Intel Core i5-10400F (6 ядер) \cite{intel};
	\item 16 Гб оперативная память DDR4;
	\item операционная система Windows 10 Pro \cite{windows}.
\end{itemize}

Во время проведения исследования компьютер был нагружен только системными приложениями и целевой программой.

\section{Время выполнения реализаций алгоритмов}

Результаты замеров времени выполнения реализаций алгоритмов умножения матриц приведены в таблицах \ref{tbl:time_measurements} -- \ref{tbl:time_measurements_odd}.
Замеры времени проводились на квадратных матрицах одного порядка и усреднялись для каждого набора одинаковых экспериментов.
В таблицах \ref{tbl:time_measurements} -- \ref{tbl:time_measurements_odd} используются следующие обозначения: 
\begin{itemize}
	\item К --- реализация классического алгоритма умножения матриц;
	\item В --- реализация алгоритма Винограда умножения матриц;
	\item ВО --- реализация алгоритма Винограда с оптимизацией умножения матриц;
	\item Ш --- реализация алгоритма Штрассена умножения матриц.
\end{itemize}

\begin{table}[h]
	\begin{center}
		\begin{threeparttable}
			\captionsetup{justification=raggedright,singlelinecheck=off}
			\caption{Время работы реализации алгоритмов (в с)}
			\label{tbl:time_measurements}
			\begin{tabular}{|c|c|c|c|c|}
				\hline
				Порядок матриц &  К  & В & ВО & Ш \\
				\hline
			6 &$ 1.875 \cdot 10^{-4} $&$ 2.031\cdot 10^{-4} $&$ 1.719\cdot 10^{-4} $&$ 1.547\cdot 10^{-3}$\\
			\hline
			11 &$ 1.047\cdot 10^{-3} $&$ 1.187\cdot 10^{-3} $&$ 8.750\cdot 10^{-4} $&$ 1.105\cdot 10^{-2}$\\
			\hline
			16 &$ 3.203\cdot 10^{-3} $&$ 3.359\cdot 10^{-3} $&$ 2.578\cdot 10^{-3} $&$ 1.095\cdot 10^{-2}$\\
			\hline
			21 &$ 7.328\cdot 10^{-3} $&$ 7.938\cdot 10^{-3} $&$ 5.656\cdot 10^{-3} $&$ 7.731\cdot 10^{-2}$\\
			\hline
			26 &$ 1.366\cdot 10^{-2} $&$ 1.406\cdot 10^{-2} $&$ 1.034\cdot 10^{-2} $&$ 7.767\cdot 10^{-2}$\\
			\hline
			31 &$ 2.289\cdot 10^{-2} $&$ 2.447\cdot 10^{-2} $&$ 1.772\cdot 10^{-2} $&$ 7.734\cdot 10^{-2}$\\
			\hline
			36 &$ 3.605\cdot 10^{-2} $&$ 3.692\cdot 10^{-2} $&$ 2.680\cdot 10^{-2} $&$ 5.358\cdot 10^{-1}$\\
			\hline
			41 &$ 5.269\cdot 10^{-2} $&$ 5.394\cdot 10^{-2} $&$ 4.081\cdot 10^{-2} $&$ 5.314\cdot 10^{-1}$\\
			\hline
			46 &$ 7.452\cdot 10^{-2} $&$ 7.644\cdot 10^{-2} $&$ 5.633\cdot 10^{-2} $&$ 5.323\cdot 10^{-1}$\\
			\hline
			51 &$ 1.003\cdot 10^{-1} $&$ 1.023\cdot 10^{-1} $&$ 7.473\cdot 10^{-2} $&$ 5.264\cdot 10^{-1}$\\
			\hline
			\end{tabular}
		\end{threeparttable}
	\end{center}
\end{table}

\begin{table}[h]
	\begin{center}
		\begin{threeparttable}
			\captionsetup{justification=raggedright,singlelinecheck=off}
			\caption{Время работы реализации алгоритмов для четных порядков матриц (в с)}
			\label{tbl:time_measurements_even}
			\begin{tabular}{|c|c|c|c|c|}
				\hline
				Порядок матриц &  К  & В & ВО & Ш \\
				\hline
				6 &$ 1.563\cdot 10^{-4} $&$ 2.188\cdot 10^{-4} $&$ 1.563\cdot 10^{-4} $&$ 1.531\cdot 10^{-3}$\\
				\hline
				10 &$ 8.125\cdot 10^{-4} $&$ 8.750\cdot 10^{-4} $&$ 6.875\cdot 10^{-4} $&$ 1.066\cdot 10^{-2}$\\
				\hline
				14 &$ 2.094\cdot 10^{-3} $&$ 2.250\cdot 10^{-3} $&$ 1.750\cdot 10^{-3} $&$ 1.097\cdot 10^{-2}$\\
				\hline
				18 &$ 4.437\cdot 10^{-3} $&$ 4.687\cdot 10^{-3} $&$ 3.531\cdot 10^{-3} $&$ 7.587\cdot 10^{-2}$\\
				\hline
				22 &$ 8.063\cdot 10^{-3} $&$ 8.406\cdot 10^{-3} $&$ 6.281\cdot 10^{-3} $&$ 7.444\cdot 10^{-2}$\\
				\hline
				26 &$ 1.325\cdot 10^{-2} $&$ 1.394\cdot 10^{-2} $&$ 1.050\cdot 10^{-2} $&$ 7.503\cdot 10^{-2}$\\
				\hline
				30 &$ 2.041\cdot 10^{-2} $&$ 2.125\cdot 10^{-2} $&$ 1.588\cdot 10^{-2} $&$ 7.553\cdot 10^{-2}$\\
				\hline
				34 &$ 2.981\cdot 10^{-2} $&$ 3.041\cdot 10^{-2} $&$ 2.244\cdot 10^{-2} $&$ 5.243\cdot 10^{-1}$\\
				\hline
				38 &$ 4.141\cdot 10^{-2} $&$ 4.216\cdot 10^{-2} $&$ 3.103\cdot 10^{-2} $&$ 5.255\cdot 10^{-1}$\\
				\hline
				42 &$ 5.688\cdot 10^{-2} $&$ 5.719\cdot 10^{-2} $&$ 4.206\cdot 10^{-2} $&$ 5.244\cdot 10^{-1}$\\
				\hline
				46 &$ 7.341\cdot 10^{-2} $&$ 7.444\cdot 10^{-2} $&$ 5.506\cdot 10^{-2} $&$ 5.239\cdot 10^{-1}$\\
				\hline
				50 &$ 9.425\cdot 10^{-2} $&$ 9.547\cdot 10^{-2} $&$ 7.016\cdot 10^{-2} $&$ 5.277\cdot 10^{-1}$\\
				\hline
			\end{tabular}
		\end{threeparttable}
	\end{center}
\end{table}

\begin{table}[h]
	\begin{center}
		\begin{threeparttable}
			\captionsetup{justification=raggedright,singlelinecheck=off}
			\caption{Время работы реализации алгоритмов для нечетных порядков матриц (в с)}
			\label{tbl:time_measurements_odd}
			\begin{tabular}{|c|c|c|c|c|}
				\hline
				Порядок матриц &  К  & В & ВО & Ш \\
				\hline
			7 &$ 2.812\cdot 10^{-4} $&$ 3.437\cdot 10^{-4} $&$ 2.812\cdot 10^{-4} $&$ 1.625\cdot 10^{-3}$\\
		\hline
		11 &$ 1.031\cdot 10^{-3} $&$ 1.219\cdot 10^{-3} $&$ 8.750\cdot 10^{-4} $&$ 1.078\cdot 10^{-2}$\\
		\hline
		15 &$ 2.563\cdot 10^{-3} $&$ 2.781\cdot 10^{-3} $&$ 2.125\cdot 10^{-3} $&$ 1.066\cdot 10^{-2}$\\
		\hline
		19 &$ 5.188\cdot 10^{-3} $&$ 5.500\cdot 10^{-3} $&$ 4.156\cdot 10^{-3} $&$ 7.531\cdot 10^{-2}$\\
		\hline
		23 &$ 9.469\cdot 10^{-3} $&$ 9.875\cdot 10^{-3} $&$ 7.250\cdot 10^{-3} $&$ 7.550\cdot 10^{-2}$\\
		\hline
		27 &$ 1.512\cdot 10^{-2} $&$ 1.584\cdot 10^{-2} $&$ 1.169\cdot 10^{-2} $&$ 7.484\cdot 10^{-2}$\\
		\hline
		31 &$ 2.266\cdot 10^{-2} $&$ 2.344\cdot 10^{-2} $&$ 1.756\cdot 10^{-2} $&$ 7.541\cdot 10^{-2}$\\
		\hline
		35 &$ 3.275\cdot 10^{-2} $&$ 3.334\cdot 10^{-2} $&$ 2.488\cdot 10^{-2} $&$ 5.311\cdot 10^{-1}$\\
		\hline
		39 &$ 4.462\cdot 10^{-2} $&$ 4.612\cdot 10^{-2} $&$ 3.466\cdot 10^{-2} $&$ 5.259\cdot 10^{-1}$\\
		\hline
		43 &$ 6.222\cdot 10^{-2} $&$ 6.256\cdot 10^{-2} $&$ 4.534\cdot 10^{-2} $&$ 5.279\cdot 10^{-1}$\\
		\hline
		47 &$ 7.841\cdot 10^{-2} $&$ 7.928\cdot 10^{-2} $&$ 6.081\cdot 10^{-2} $&$ 5.288\cdot 10^{-1}$\\
		\hline
		51 &$ 1.037\cdot 10^{-1} $&$ 1.059\cdot 10^{-1} $&$ 7.500\cdot 10^{-2} $&$ 5.279\cdot 10^{-1}$\\
			\hline
			\end{tabular}
		\end{threeparttable}
	\end{center}
\end{table}

\clearpage
На рисунках \ref{img:figure} -- \ref{img:figureOdd} изображены графики зависимостей времени выполнения реализаций алгоритмов от порядка умножаемых матриц.

\includeimage
{figure} % Имя файла без расширения (файл должен быть расположен в директории inc/img/)
{f} % Обтекание (без обтекания)
{h} % Положение рисунка (см. figure из пакета float)
{1\textwidth} % Ширина рисунка
{Сравнение реализаций алгоритмов по времени выполнения} % Подпись рисунка

\includeimage
{figure_even} % Имя файла без расширения (файл должен быть расположен в директории inc/img/)
{f} % Обтекание (без обтекания)
{h} % Положение рисунка (см. figure из пакета float)
{1\textwidth} % Ширина рисунка
{Сравнение реализаций алгоритмов по времени выполнения для четных порядков матриц} % Подпись рисунка

\includeimage
{figure_odd} % Имя файла без расширения (файл должен быть расположен в директории inc/img/)
{f} % Обтекание (без обтекания)
{h} % Положение рисунка (см. figure из пакета float)
{1\textwidth} % Ширина рисунка
{Сравнение реализаций алгоритмов по времени выполнения для нечетных порядков матриц} % Подпись рисунка


\clearpage

\section*{Вывод}

В результате замеров времени выполнения реализаций различных алгоритмов было получено, что для матриц с порядком 51, реализация алгоритма Винограда оказалась в 1.02 раза хуже реализации классического алгоритма по времени выполнения, при этом реализация алгоритма Винограда с оптимизациями оказалась лучше в 1.34 раза реализации классического алгоритма и в 1.37 раз лучше реализации без оптимизаций по времени выполнения. 
Для матриц с четным порядком 46, реализация алгоритма Винограда хуже реализации классического алгоритма в 1.02 раза по времени выполнения, а реализация алгоритма Винограда с оптимизацией оказалась лучше классического алгоритма в 1.32 раза. 

Для матриц с порядком 51, реализация алгоритма Штрассена оказалась хуже по времени выполнения в 5.24 раза, в 5.14 раза и в 7.04 раза, чем реализации классического алгоритма, алгоритма Винограда и алгоритма Винограда с оптимизациями соответственно. 
Такой результат обусловлен тем, что в данной реализации много времени тратится на выделение подматриц, их сложение и вычитание, а также на рекурсивные вызовы. По теоретической оценке трудоемкости, можно заметить насколько велика константа перед $n^{\log_{2}7}$.

Стоит заметить, что на графике зависимости времени выполнения реализаций алгоритмов от порядка матриц \ref{img:figure}, у реализации алгоритма Штрассена присутствуют отчетливые ступеньки.
Это обусловлено особенностью алгоритма, который может работать только с квадратными матрицами порядка $2^k$, а матрицы других размеров приводятся к данному размеру дополнением нулевыми строками и столбцами. 
Таким образом, разные размеры $n$ могу приводится к одному новому порядку $2^k$, что в свою очередь приводит к почти идентичному времени обработки.

Полученные на практике результаты примерно соответствуют теоретическим оценкам.