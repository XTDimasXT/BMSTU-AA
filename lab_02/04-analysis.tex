\chapter{Аналитический раздел}

Матрицей называется прямоугольная таблица чисел, вида \eqref{eq:matrix}, состоящая из $m$ строк и $n$ столбцов~\cite{matrix}.

\begin{equation}
	\label{eq:matrix}
	\begin{pmatrix}
		a_{11} & a_{12} & \ldots & a_{1n}\\
		a_{21} & a_{22} & \ldots & a_{2n}\\
		\vdots & \vdots & \ddots & \vdots\\
		a_{m1} & a_{m2} & \ldots & a_{mn}
	\end{pmatrix},
\end{equation}

Пусть $A$ --- матрица, тогда $а_{ij}$ --- элемент этой матрицы, который находится на \textit{i-ой} строке и \textit{j-ом} столбце.

Если количество столбцов первой матрицы совпадает с количеством строк второй матрицы, то возможно выполнить их матричное умножение.
В результате умножения получится матрица-произведение, количество строк в которой равно количеству строк первой матрицы, а количество столбцов равно количеству столбцов второй матрицы.

\section{Классический алгоритм}

Пусть даны две прямоугольные матрицы $A$ и $B$ размеров $[m \times n]$ и $[n \times k]$ соответственно. В результате произведение матриц $A$ и $B$ получим матрицу $C$ размера $[m \times k]$, элементы которой вычисляются по следующей формуле

\begin{equation}
	\label{eq:matrix_classic}
	c_{ij} = \sum_{l=1}^{n}a_{il}b_{lj}
\end{equation}

Классический алгоритм умножения матриц реализует формулу \eqref{eq:matrix_classic}.

\section{Алгоритм Винограда}

Произведение матриц $A$ и $B$ можно связать со скалярным  произведением строки матрицы $A$ на столбец матрицы $B$ \cite{book_vinograd}.

Рассмотрим два вектора $V = (v1, v2, v3, v4)$ и $W = (w1, w2, w3, w4)$.  

Их скалярное произведение равно

\begin{equation} \label{formula}
	V \cdot W=v_1 \cdot w_1 + v_2 \cdot w_2 + v_3 \cdot w_3 + v_4 \cdot w_4
\end{equation}

Равенство (\ref{formula}) можно переписать в виде
\begin{equation} \label{formula2}
	V \cdot W=(v_1 + w_2) \cdot (v_2 + w_1) + (v_3 + w_4) \cdot (v_4 + w_3) - v_1 \cdot v_2 - v_3 \cdot v_4 - w_1 \cdot w_2 - w_3 \cdot w_4
\end{equation}

Теперь допустим, что у нас есть две матрицы $A$ и $B$ размерности $m \times n$ и $n \times p$ соответственно, и мы хотим найти их произведение $C = A \cdot B$.
Тогда алгоритм будет состоять из следующих шагов.

\begin{enumerate}
	\item Подготовительные вычисления. Сначала создаются два вспомогательных массива
	
	\begin{equation}
	\label{eq:rf}
	\text{{rowFactor}}[i] = \sum_{j=0}^{n/2 - 1} A[i][2j+1] \cdot A[i][2j]
	\end{equation}
	для \(0 \leq i < m\)
	\begin{equation}
	\label{eq:cf}
	\text{{colFactor}}[j] = \sum_{i=0}^{n/2 - 1} B[2i+1][j] \cdot B[2i][j]
	\end{equation}
	для \(0 \leq j < p\).
	
	\item Умножение матриц. Вычисляем результирующую матрицу $C$ по формуле
	\begin{equation}
		\label{eq:c_res}
		\begin{aligned}
			C[i][j] &= \sum_{k=0}^{n/2 - 1} (A[i][2k+1] + B[2k][j]) \cdot (A[i][2k] + B[2k+1][j]) \\
			&\quad - \text{{rowFactor}}[i] - \text{{colFactor}}[j]
		\end{aligned}
	\end{equation}
	для \(0 \leq i < m\) и \(0 \leq j < p\).
	
	\item Коррекция. Если \(n\) нечетно, добавляем коррекцию, в соответствии со следующей формулой
	
	\begin{equation}
		\label{eq:er}
	C[i][j] += A[i][n] \cdot B[n][j]
	\end{equation}
	для \(0 \leq i < m\) и \(0 \leq j < p\). 
\end{enumerate}

\section{Алгоритм Штрассена}

Алгоритм Штрассена --- это алгоритм умножения квадратных матриц, который является более эффективным для больших матриц, чем классический метод умножения~\cite{strassen}.

Если добавить к матрицам $A$ и $B$ одинаковые нулевые строки и столбцы, их произведение станет равно матрице $C$ с теми же добавленными строками и столбцами. 
Поэтому в данном алгоритме рассматриваются матрицы порядка $2^{k + 1}$, где $ k \in \mathbb{N} $, а все остальные матрицы, сводятся к этому размеру добавлением нулевых строк и столбцов. 

Алгоритм состоит из следующих шагов.

\begin{enumerate}
	\item Разбиение матриц
	\begin{equation}
		\label{eq:mat_split}
	A = \begin{bmatrix} A_{11} & A_{12} \\ A_{21} & A_{22} \end{bmatrix}, \quad
	B = \begin{bmatrix} B_{11} & B_{12} \\ B_{21} & B_{22} \end{bmatrix}
	,
	\end{equation} где $A_{ij}$ и $B_{ij}$ -- матрицы порядка $2^{k}$
	
	\item Вычисление вспомогательных матриц
	\begin{equation}
		\label{eq:help_mat}
	\begin{aligned}
		M_1 &= (A_{11} + A_{22})(B_{11} + B_{22}) \\
		M_2 &= (A_{21} + A_{22})B_{11} \\
		M_3 &= A_{11}(B_{12} - B_{22}) \\
		M_4 &= A_{22}(B_{21} - B_{11}) \\
		M_5 &= (A_{11} + A_{12})B_{22} \\
		M_6 &= (A_{21} - A_{11})(B_{11} + B_{12}) \\
		M_7 &= (A_{12} - A_{22})(B_{21} + B_{22})
	\end{aligned}
	\end{equation}
	
	\item Вычисление результирующих подматриц
	\begin{equation}
		\label{eq:res_submat}
	\begin{aligned}
		C_{11} &= M_1 + M_4 - M_5 + M_7 \\
		C_{12} &= M_3 + M_5 \\
		C_{21} &= M_2 + M_4 \\
		C_{22} &= M_1 - M_2 + M_3 + M_6
	\end{aligned}
	\end{equation}
		
	Результирующая матрица состоит из $C_{ij}$
	\begin{equation}
		\label{eq:res_mat_str}
		AB = C = \begin{bmatrix} C_{11} & C_{12} \\ C_{21} & C_{22} \end{bmatrix}
	\end{equation}
		
\end{enumerate}


\section*{Вывод}
В данном разделе были описаны алгоритмы классического умножения матриц, алгоритм Штрассена и алгоритм Винограда.